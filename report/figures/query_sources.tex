\begin{figure}[h]
    \centering
    \subfloat[Requête exécutée sur la fédération $E1,E2,E3,E4$]{
        \lstinputlisting[basicstyle=\scriptsize\sffamily,language=sparql,numbers=none,frame=none,columns=fixed,extendedchars=true,breaklines=true,showstringspaces=false]{sparql/queryExample.txt}
         \label{query:movies}
     }
    \subfloat[Fragments (F) \& Sélection de source par \fedra]{
    \label{fig:source_selection_movies}
        \begin{scriptsize}
            \begin{tabular}{|c|c|c|c|} \hline
                Triple pattern & F & Endpoints & Sélection de sources \\
                \hline
                tp1 & f3 & E4 & $\{ \{ E4 \} \}$ \\
                \hline
                & f1 & E1 &  \\
                \cline{3-3}
                tp2 &  & E2 & $\{\{ E1,E2 \} , E3 \}$ \\
                 \cline{2-3}
                 & f2 & E3 &  \\
                \hline
            \end{tabular}
        \end{scriptsize}
    }
    \caption{Exemple de sélection de sources effectuée par \fedra pour une requête SPARQL fédérée}
\end{figure}